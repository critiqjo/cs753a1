%%%%%%%%%%%%%%%%%%%%%%
%% Template adapted from: 
%% https://www.overleaf.com/latex/examples/introduction-to-electrical-engineering-example-assignment-template/pqvbrbjtjcqq#.WH72rTXWBC0
%%%%%%%%%%%%%%%%%%%%%%

\documentclass[11pt,a4paper,titlepage]{article}
\usepackage[a4paper]{geometry}
\usepackage[utf8]{inputenc}
\usepackage[english]{babel}
\usepackage{booktabs}

\usepackage{amsmath, amssymb, amsfonts, amsthm, mathtools}
\usepackage{microtype} %improves the spacing between words and letters
\usepackage{graphicx}


% ------------------------------------------------
%% COLOR DEFINITIONS
% ------------------------------------------------
\usepackage[svgnames]{xcolor} % Enabling mixing colors and color's call by 'svgnames'
% ------------------------------------------------
\definecolor{MyBlue}{rgb}{0.2,0.4,0.6} %mix personal color
\newcommand{\blue}{\color{MyBlue} \usefont{OT1}{lmss}{m}{n}}
\newcommand{\blueb}{\color{MyBlue} \usefont{OT1}{lmss}{b}{n}}
% ------------------------------------------------


% ------------------------------------------------
%% FONTS AND COLORS
% ------------------------------------------------
\usepackage{sectsty}
%set section/subsections HEADINGS font and color
\sectionfont{\color{MyBlue}}  % sets colour of sections
\subsectionfont{\color{MyBlue}}  % sets colour of sections

\usepackage{hyperref}
\hypersetup{colorlinks=true}

% ------------------------------------------------
%% TITLE
% ------------------------------------------------
\title{\blue CS 753: Automatic Speech Recognition \\ \blueb Assignment 1}
\author{John C F \\ \small 14305R006}
\date{\today}
% ------------------------------------------------

\begin{document}
\maketitle
%\tableofcontents{}
\newpage

% ------------------------------------------------

\section*{Problem 1 Solution}
\subsection*{Part (a)}

\addtolength{\jot}{0.25em}
\begin{enumerate}
    \item (i) To prove $\Pr(A | B) > \Pr(A) \iff \Pr(B | A) > \Pr(B)$.
        \begin{align*}
            \Pr(A | B) &> \Pr(A) \\
            \frac{\Pr(A \cap B)}{\Pr(B)} &> \Pr(A) \\
            \frac{\Pr(A \cap B)}{\Pr(A)} &> \Pr(B) \\
            \therefore~\Aboxed{\Pr(B | A) &> \Pr(A)}
        \end{align*}

        (ii) To prove $\Pr(A | B) > \Pr(A) \iff \Pr(A^c | B^c) > \Pr(A^c)$.
        \begin{align*}
            \Pr(A | B) &> \Pr(A) \\
            \frac{\Pr(A \cap B)}{\Pr(B)} &> \Pr(A) \\
            \Pr(A \cap B) &> \Pr(A) \Pr(B) \\
            1 - \Pr(A \cap B) &< 1 - \Pr(A) \Pr(B) \\
            \Pr((A \cap B)^c) &< 1 - (1 - \Pr(A^c)) (1 - \Pr(B^c)) \\
            \Pr(A^c \cup B^c) &< 1 - (1 - \Pr(A^c) - \Pr(B^c) + \Pr(A^c) \Pr(B^c)) \\
            \Pr(A^c) + \Pr(B^c) - \Pr(A^c \cap B^c) &< \Pr(A^c) + \Pr(B^c) - \Pr(A^c) \Pr(B^c) \\
            \Pr(A^c \cap B^c) &> \Pr(A^c) \Pr(B^c) \\
            \frac{\Pr(A^c \cap B^c)}{\Pr(B^c)} &> \Pr(A^c) \\
            \therefore~\Aboxed{\Pr(A^c | B^c) &> \Pr(A^c)}
        \end{align*}

    \item Let $S_b$ denote sending bit $b$, and $R_b$ denote receiving bit $b$.
        The problem is to prove that $\Pr(S_0 | R_0) > \Pr(S_0)$ given that the
        probability of the bit being flipped by the channel is $\epsilon < 1/2$.
        Given:
        \begin{align*}
            \Pr(S_0) &= p & \Pr(R_0 | S_0) &= 1 - \epsilon \\
            \Pr(S_1) &= 1 - p & \Pr(R_0 | S_1) &= \epsilon \\
        \end{align*}
        Now,
        \begin{align}
            \Pr(S_0 | R_0) &= \frac{\Pr(R_0 \cap S_0)}{\Pr(R_0)} \nonumber \\
                           &= \frac{\Pr(R_0 | S_0) \Pr(S_0)}
                                   {\Pr(R_0 | S_0)\Pr(S_0) + \Pr(R_0 | S_1)\Pr(S_1)} \nonumber \\
                           &= \frac{(1 - \epsilon) p}
                                   {(1 - \epsilon) p + \epsilon (1 - p)} \nonumber \\
                           &= p \cdot \frac{1}{p + \frac{\epsilon}{1 - \epsilon} \cdot (1 - p)} \label{eq:s0r0} \\
                           &< p \nonumber
        \end{align}
        Because
        \begin{align*}
                        &\epsilon < 1/2 \\
            \Rightarrow~&\frac{\epsilon}{1 - \epsilon} < 1 \\
            \Rightarrow~&p + \frac{\epsilon}{1 - \epsilon} \cdot (1 - p) < 1 \\
            \Rightarrow~& \Pr(S_0 | R_0) > p \tag{from \ref{eq:s0r0}} \\
            \therefore~\Aboxed{&\Pr(S_0 | R_0) > \Pr(S_0)}
        \end{align*}
\end{enumerate}


\subsection*{Part (b)}

If $p_k$ is the probability that the ant gets the sugar if it starts $k$
centimeters to the right of the sugar, then our aim is to find $p_1$.
Now, for $0 < k \le n$,
\begin{align}
    p_0 &= 1 \nonumber \\
    p_{n+1} &= 0 \label{eq:p_n1} \\
    p_k &= 0.5 p_{k-1} + 0.5 p_{k+1} \label{eq:p_k}
\end{align}

For $k = n$ in equation \ref{eq:p_k},
\begin{align}
    p_n &= \frac{1}{2} p_{n-1} \label{eq:p_n}
\end{align}

For $k = n - 1$ in equation \ref{eq:p_k} and $n > 1$,
\begin{align*}
    p_{n-1} &= \frac{1}{2} p_{n-2} + \frac{1}{2} p_n \\
            &= \frac{1}{2} p_{n-2} + \frac{1}{4} p_{n-1} \tag{from \ref{eq:p_n}} \\
    \Rightarrow \frac{3}{4} p_{n-1} &= \frac{1}{2} p_{n-2} \\
    \Rightarrow p_{n-1} &= \frac{2}{3} p_{n-2}
\end{align*}

Similarly, for $k = n - 2$ and $n > 2$,
\begin{align*}
    p_{n-2} &= \frac{1}{2} p_{n-3} + \frac{1}{2} p_{n-1} \\
    \Rightarrow p_{n-2} &= \frac{3}{4} p_{n-3}
\end{align*}

And for $k = n - 3$ and $n > 3$, we get
\begin{align*}
    p_{n-3} &= \frac{4}{5} p_{n-4}
\end{align*}

This pattern can be generalized as, for $n > c \ge 0$,
\begin{equation}\label{eq:pnk_rec}
    p_{n-c} = \frac{c + 1}{c + 2} \cdot p_{n-c-1}
\end{equation}

For $c = n - 2$ in equation \ref{eq:pnk_rec} and $n \ge 2$,
\begin{align}
    p_2 &= \frac{n - 1}{n} p_1 \label{eq:p_2}
\end{align}

For $k = 1$ in equation \ref{eq:p_k},
\begin{align}
    p_1 &= \frac{1}{2} p_0 + \frac{1}{2} p_2 \nonumber \\
    \Rightarrow 2 p_1 &= 1 + p_2 \label{eq:p_1}
\end{align}

Equation \ref{eq:p_2} in equation \ref{eq:p_1}, for $n \ge 2$,
\begin{align*}
    2p_1 &= 1 + \frac{n - 1}{n} p_1 \\
    \Rightarrow \frac{n + 1}{n} p_1 &= 1 \\
    \Rightarrow~\Aboxed{p_1 &= \frac{n}{n + 1}}
\end{align*}

For $n = 1$, $p_1 = 1/2$, since the ant can only take a single random move.
Thus, the above equation holds for $n \ge 1$.

% ------------------------------------------------

\section*{Problem 2 Solution}

\subsection*{Part (c)}

Normalize the weights given due to unigram frequencies to be between 0 and 1.
This way, edit distance will always take precedence.

\subsection*{Part (d)}

\begin{itemize}
    \item Lower the cost of changing letters that sounds similar (e.g. c to k,
        o to au etc.).
    \item Lower the cost of deletion of one of the consonants that are next to
        each other in QWERTY layout, when appearing together (a mistype).
    \item Lower the cost of repeating a consonant when in the middle of a word
        (esp. t to tt).
\end{itemize}

% ------------------------------------------------

\section*{Problem 3 Solution}
\subsection*{Part (a)}
(i) To calculate: $\Pr(q_{t+1} = k | q_t = j, o_1, \cdots o_T)$
\begin{align*}
    ~~ &= \Pr(q_{t+1} = k | q_t = j, o_{t+1}, \cdots o_T) \tag{Markov property} \\
       &= \frac{\Pr(q_{t+1} = k, o_1, \cdots o_T | q_t = j)}
               {\Pr(o_{t+1}, \cdots o_T | q_t = j)} \\
       &= \frac{\Pr(o_{t+1}, \cdots o_T | q_{t+1} = k, q_t = j) \cdot \Pr(q_{t+1} = k | q_t = j)}
               {\beta_t(j)} \\
       &= \frac{\Pr(o_{t+1}, \cdots o_T | q_{t+1} = k) \cdot a_{jk}}
               {\beta_t(j)} \tag{Markov prop.} \\
       &= \frac{\Pr(o_{t+2}, \cdots o_T | o_{t+1}, q_{t+1} = k) \cdot \Pr(o_{t+1} | q_{t+1} = k) \cdot a_{jk}}
               {\beta_t(j)} \\
       &= \frac{\Pr(o_{t+2}, \cdots o_T | q_{t+1} = k) \cdot b_k(o_{t+1}) \cdot a_{jk}}
               {\beta_t(j)} \tag{Markov prop.} \\
       &= \frac{\beta_{t+1}(k) \cdot b_k(o_{t+1}) \cdot a_{jk}}
               {\beta_t(j)}
\end{align*}
For the sake of next part, let's define a function $p_t(i, j)$ as follows:
\begin{align*}
    p_t(i, j) &= \Pr(q_{t+1} = j | q_t = i, o_1, \cdots o_T) \\
              &= a_{ij} \cdot b_j(o_{t+1}) \cdot \frac{\beta_{t+1}(j)}{\beta_t(i)}
\end{align*}
(ii) To calculate: $\Pr(q_{t-1} = i, q_t = j, q_{t+1} = k | o_1, \cdots o_T)$
\begin{align*}
    ~~ &= \Pr(q_{t+1} = k | q_{t-1} = i, q_t = j, o_1, \cdots o_T) \cdot \Pr(q_{t-1} = i, q_t = j | o_1, \cdots o_T) \\
       &= \Pr(q_{t+1} = k | q_t = j, o_1, \cdots o_T) \cdot \Pr(q_{t-1} = i, q_t = j | o_1, \cdots o_T) \tag{Markov prop.} \\
       &= p_t(j, k) \cdot \Pr(q_t = j | q_{t-1} = i, o_1, \cdots o_T) \cdot \Pr(q_{t-1} = i | o_1, \cdots o_T) \\
       &= p_t(j, k) \cdot p_{t-1}(i, j) \cdot \frac{\Pr(q_{t-1} = i, o_1, \cdots o_T)}{\Pr(o_1, \cdots o_T)} \\
       &= p_t(j, k) \cdot p_{t-1}(i, j) \cdot
          \frac{\Pr(o_t, \cdots o_T | o_1, \cdots o_{t-1}, q_{t-1} = i) \cdot
                \Pr(o_1, \cdots o_{t-1}, q_{t-1} = i)}
               {\sum_{h \in Q}\Pr(o_1, \cdots o_T | q_T = h) \cdot \Pr(q_T = h)} \\
       &= p_t(j, k) \cdot p_{t-1}(i, j) \cdot
          \frac{\Pr(o_t, \cdots o_T | q_{t-1} = i) \cdot \alpha_{t-1}(i)}
               {\sum_{h \in Q}\Pr(o_1, \cdots o_T, q_T = h)} \\
       &= p_t(j, k) \cdot p_{t-1}(i, j) \cdot
          \frac{\beta_{t-1}(i) \cdot \alpha_{t-1}(i)}{\sum_{h \in Q}\alpha_T(h)}
\end{align*}
Now, expanding $p_t(j, k)$ and $p_{t-1}(i, j)$,
\begin{align*}
    ~~ &= a_{jk} \cdot b_k(o_{t+1}) \cdot \frac{\beta_{t+1}(k)}{\beta_t(j)} \cdot
          a_{ij} \cdot b_j(o_t) \cdot \frac{\beta_t(j)}{\beta_{t-1}(i)} \cdot
          \frac{\beta_{t-1}(i) \cdot \alpha_{t-1}(i)}{\sum_{h \in Q}\alpha_T(h)} \\
       &= a_{ij} \cdot a_{jk} \cdot b_k(o_{t+1}) \cdot b_j(o_t) \cdot
          \frac{\beta_{t+1}(k) \cdot \alpha_{t-1}(i)}{\sum_{h \in Q}\alpha_T(h)} \\
\end{align*}

\subsection*{Part (b)}

\textbf{Recursion:} For all $1 < t \le T$,
\begin{align*}
    v_{t-1, max} &= \max_{j=1}^N v_{t-1}(j) \\
    bt_{t-1, max} &= \textrm{arg}\max_{j=1}^N v_{t-1}(j) % FIXME
\end{align*}
And for all $1 \le j \le N$,
\begin{align*}
    v_t(j) &=
    \begin{dcases}
        v_{t-1, max} \; q \; b_j(o_t)
            & {\rm if~} bt_{t-1, max} \ne j {\rm~and~} v_{t-1, max} \; q > v_{t-1}(j) \; p \\
        v_{t-1}(j) \; p \; b_j(o_t)
            & \textrm{otherwise}
    \end{dcases} \\
    bt_t(j) &=
    \begin{dcases}
        bt_{t-1,max}
            & {\rm if~} bt_{t-1, max} \ne j {\rm~and~} v_{t-1, max} \; q > v_{t-1}(j) \; p \\
        j
            & \textrm{otherwise}
    \end{dcases}
\end{align*}

\noindent \textbf{Initialization} and \textbf{Termination} steps remain unchanged.

\subsection*{Part (c)}

\begin{enumerate}
    \item \textbf{Initialization:} For all $1 \le j \le N$,
        \begin{align*}
            v_1(j) &= a_{0j} \; b_j(o1) \\
            bt_1(j) &= 0 \\
            c_1(j) &= 0
        \end{align*}
    \item \textbf{Recursion:} For all $1 < t \le T$ and $1 \le j \le N$,\\
        If $c_{t-1}(j) < k$,
        \begin{align*}
            v_t(j) &= \max_{i=1}^{N} v_{t-1}(i) \; a_{ij} \; b_j(ot) \\
            bt_t(j) &= \textrm{arg}\max_{i=1}^{N} v_{t-1}(i) \; a_{ij} \; b_j(ot) \\
            c_t(j) &=
            \begin{dcases}
                c_{t-1}(j) + 1 & {\rm if~} bt_t(j) = j \\
                0 & {\rm otherwise} \\
            \end{dcases}
        \end{align*}
        otherwise,
        \begin{align*}
            v_t(j) &= \max_{1 \le i \le N}^{i \ne j} v_{t-1}(i) \; a_{ij} \; b_j(ot) \\
            bt_t(j) &= \textrm{arg}\max_{1 \le i \le N}^{i \ne j} v_{t-1}(i) \; a_{ij} \; b_j(ot) \\
            c_t(j) &= 0
        \end{align*}
    \item \textbf{Termination:} (unchanged)
        \begin{align*}
            \textrm{The best score:}&~~ P* = v_T(q_F) = \max_{i=1}^N v_T(i) \; a_{iF} \\
            \textrm{The start of backtrace:}&~~ q_T* = bt_T(q_F) = {\rm arg}\max_{i=1}^N v_T(i) \; a_{iF}
        \end{align*}
\end{enumerate}

% ------------------------------------------------

\section*{Problem 4 Solution}
\subsection*{Part (a)}

\end{document}
