%%%%%%%%%%%%%%%%%%%%%%
%% Template adapted from: 
%% https://www.overleaf.com/latex/examples/introduction-to-electrical-engineering-example-assignment-template/pqvbrbjtjcqq#.WH72rTXWBC0
%%%%%%%%%%%%%%%%%%%%%%

\documentclass[11pt,a4paper,titlepage]{article}
\usepackage[a4paper]{geometry}
\usepackage[utf8]{inputenc}
\usepackage[english]{babel}
\usepackage{booktabs}

\usepackage{amsmath, amssymb, amsfonts, amsthm, mathtools}
\usepackage{microtype} %improves the spacing between words and letters
\usepackage{graphicx}


% ------------------------------------------------
%% COLOR DEFINITIONS
% ------------------------------------------------
\usepackage[svgnames]{xcolor} % Enabling mixing colors and color's call by 'svgnames'
% ------------------------------------------------
\definecolor{MyBlue}{rgb}{0.2,0.4,0.6} %mix personal color
\newcommand{\blue}{\color{MyBlue} \usefont{OT1}{lmss}{m}{n}}
\newcommand{\blueb}{\color{MyBlue} \usefont{OT1}{lmss}{b}{n}}
% ------------------------------------------------


% ------------------------------------------------
%% FONTS AND COLORS
% ------------------------------------------------
\usepackage{sectsty}
%set section/subsections HEADINGS font and color
\sectionfont{\color{MyBlue}}  % sets colour of sections
\subsectionfont{\color{MyBlue}}  % sets colour of sections

\usepackage{hyperref}
\hypersetup{colorlinks=true}

% ------------------------------------------------
%% TITLE
% ------------------------------------------------
\title{\blue CS 753: Automatic Speech Recognition \\ \blueb Assignment 1 (Part 1)}
\author{John C F \\ \small 14305R006}
\date{\today}
% ------------------------------------------------

\begin{document}
\maketitle
%\tableofcontents{}
\newpage

% ------------------------------------------------

\section{Problem 1 Solution}
\subsection*{Part (a)}

\addtolength{\jot}{0.25em}
\begin{enumerate}
    \item (i) To prove $\Pr(A | B) > \Pr(A) \iff \Pr(B | A) > \Pr(B)$.
        \begin{align*}
            \Pr(A | B) &> \Pr(A) \\
            \frac{\Pr(A \cap B)}{\Pr(B)} &> \Pr(A) \\
            \frac{\Pr(A \cap B)}{\Pr(A)} &> \Pr(B) \\
            \therefore~\Aboxed{\Pr(B | A) &> \Pr(A)}
        \end{align*}

        (ii) To prove $\Pr(A | B) > \Pr(A) \iff \Pr(A^c | B^c) > \Pr(A^c)$.
        \begin{align*}
            \Pr(A | B) &> \Pr(A) \\
            \frac{\Pr(A \cap B)}{\Pr(B)} &> \Pr(A) \\
            \Pr(A \cap B) &> \Pr(A) \Pr(B) \\
            1 - \Pr(A \cap B) &< 1 - \Pr(A) \Pr(B) \\
            \Pr((A \cap B)^c) &< 1 - (1 - \Pr(A^c)) (1 - \Pr(B^c)) \\
            \Pr(A^c \cup B^c) &< 1 - (1 - \Pr(A^c) - \Pr(B^c) + \Pr(A^c) \Pr(B^c)) \\
            \Pr(A^c) + \Pr(B^c) - \Pr(A^c \cap B^c) &< \Pr(A^c) + \Pr(B^c) - \Pr(A^c) \Pr(B^c) \\
            \Pr(A^c \cap B^c) &> \Pr(A^c) \Pr(B^c) \\
            \frac{\Pr(A^c \cap B^c)}{\Pr(B^c)} &> \Pr(A^c) \\
            \therefore~\Aboxed{\Pr(A^c | B^c) &> \Pr(A^c)}
        \end{align*}

    \item Let $S_b$ denote sending bit $b$, and $R_b$ denote receiving bit $b$.
        The problem is to prove that $\Pr(S_0 | R_0) > \Pr(S_0)$ given that the
        probability of the bit being flipped by the channel is $\epsilon < 1/2$.
        Given:
        \begin{align*}
            \Pr(S_0) &= p & \Pr(R_0 | S_0) &= 1 - \epsilon \\
            \Pr(S_1) &= 1 - p & \Pr(R_0 | S_1) &= \epsilon \\
        \end{align*}
        Now,
        \begin{align}
            \Pr(S_0 | R_0) &= \frac{\Pr(R_0 \cap S_0)}{\Pr(R_0)} \nonumber \\
                           &= \frac{\Pr(R_0 | S_0) \Pr(S_0)}
                                   {\Pr(R_0 | S_0)\Pr(S_0) + \Pr(R_0 | S_1)\Pr(S_1)} \nonumber \\
                           &= \frac{(1 - \epsilon) p}
                                   {(1 - \epsilon) p + \epsilon (1 - p)} \nonumber \\
                           &= p \cdot \frac{1}{p + \frac{\epsilon}{1 - \epsilon} \cdot (1 - p)} \label{eq:s0r0} \\
                           &< p \nonumber
        \end{align}
        Because
        \begin{align*}
                        &\epsilon < 1/2 \\
            \Rightarrow~&\frac{\epsilon}{1 - \epsilon} < 1 \\
            \Rightarrow~&p + \frac{\epsilon}{1 - \epsilon} \cdot (1 - p) < 1 \\
            \Rightarrow~& \Pr(S_0 | R_0) > p \tag{from \ref{eq:s0r0}} \\
            \therefore~\Aboxed{&\Pr(S_0 | R_0) > \Pr(S_0)}
        \end{align*}
\end{enumerate}


\subsection*{Part (b)}

If $p_k$ is the probability that the ant gets the sugar if it starts $k$
centimeters to the right of the sugar, then our aim is to find $p_1$.
Now, for $0 < k \le n$,
\begin{align}
    p_0 &= 1 \nonumber \\
    p_{n+1} &= 0 \label{eq:p_n1} \\
    p_k &= 0.5 p_{k-1} + 0.5 p_{k+1} \label{eq:p_k}
\end{align}

For $k = n$ in equation \ref{eq:p_k},
\begin{align}
    p_n &= \frac{1}{2} p_{n-1} \label{eq:p_n}
\end{align}

For $k = n - 1$ in equation \ref{eq:p_k} and $n > 1$,
\begin{align*}
    p_{n-1} &= \frac{1}{2} p_{n-2} + \frac{1}{2} p_n \\
            &= \frac{1}{2} p_{n-2} + \frac{1}{4} p_{n-1} \tag{from \ref{eq:p_n}} \\
    \Rightarrow \frac{3}{4} p_{n-1} &= \frac{1}{2} p_{n-2} \\
    \Rightarrow p_{n-1} &= \frac{2}{3} p_{n-2}
\end{align*}

Similarly, for $k = n - 2$ and $n > 2$,
\begin{align*}
    p_{n-2} &= \frac{1}{2} p_{n-3} + \frac{1}{2} p_{n-1} \\
    \Rightarrow p_{n-2} &= \frac{3}{4} p_{n-3}
\end{align*}

And for $k = n - 3$ and $n > 3$, we get
\begin{align*}
    p_{n-3} &= \frac{4}{5} p_{n-4}
\end{align*}

This pattern can be generalized as, for $n > c \ge 0$,
\begin{equation}\label{eq:pnk_rec}
    p_{n-c} = \frac{c + 1}{c + 2} \cdot p_{n-c-1}
\end{equation}

For $c = n - 2$ in equation \ref{eq:pnk_rec} and $n \ge 2$,
\begin{align}
    p_2 &= \frac{n - 1}{n} p_1 \label{eq:p_2}
\end{align}

For $k = 1$ in equation \ref{eq:p_k},
\begin{align}
    p_1 &= 0.5 p_0 + 0.5 p_2 \nonumber \\
        &= 0.5 + 0.25 p_1 + 0.25 p_2 \nonumber \\
    \Rightarrow \frac{3}{4} p_1 &= \frac{1}{2} + \frac{1}{4} p_2 \nonumber \\
    \Rightarrow p_1 &= \frac{2}{3} + \frac{1}{3} p_2 \label{eq:p_1}
\end{align}

Equation \ref{eq:p_2} in equation \ref{eq:p_1}, for $n \ge 2$,
\begin{align*}
    p_1 &= \frac{2}{3} + \frac{n - 1}{3n} p_1 \\
    \Rightarrow \frac{2n + 1}{3n} p_1 &= \frac{2}{3} \\
    \Rightarrow p_1 &= \frac{2n}{2n + 1}
\end{align*}

% ------------------------------------------------

\section{Problem 2 Solution}
\subsection*{Part (a)}

\subsection*{Part (b)}

% ------------------------------------------------

\section{Problem 3 Solution}
\subsection*{Part (a)}

\subsection*{Part (b)}
Replacing capacitor with a test voltage, as shown on 
\begin{align}
v_t-v_{\phi}&=25k\alpha \cdot v_{\phi} \nonumber \\
v_t=(1+25k\alpha) \cdot v_{\phi}&=15k(1+25k\alpha) \cdot i_t \nonumber \\
\Aboxed{R_{in}(C)&=15k(1+25k\alpha) \label{eq:10}}
\end{align}

\subsection*{Part (c)}
\begin{align*}
\Aboxed{\alpha \approx 2.26 \cdot 10^{-4}}
\end{align*}

% ------------------------------------------------

\end{document}
